\section{IP YÖNLENDİRME}

Ip'nın yönlendirilebilir olmasi protokolün  en güçlü özelliği çok sayıda iletişim protokolu mevcut olmasına rağmen IP'nin yönlendirilebilir esnek yapısı internetin temel dili olmasını sağlamıştır.Yönlednirme işlemini "Yönlendirici(Router) "yapar.
\textbf{Yönlendirme Tablosunda}\\
\begin{enumerate}[label=\alph*)] 
   \item Kaynak
   \item Hedef(Ip ve Maske)
   \item Ağ Geçidi
   \item Ara Birim(Interface)
   \item Ölçüt(Metrik)
\end{enumerate}

\begin{figure}[ht]
    \centering
    \includegraphics[width=17cm]{images/yonlendirme.png}
    \caption{Yönlendirme Tablosu}
    \label{fig:bandwidth_example}
\end{figure}

\subsection{STATİK YÖNLENDİRME}
Ağ yöneticisi tarafından elle sabit olarak yazılır.Genellikle yönlendiricisi ve yönlendirme işlemi çok fazla olmayan ağlarda kullanılır.Yönlendirme tablolarının güncellemesi ağdaki fiziksel değişikliklere göre yeniden elle yapılmalıdır.\\
\begin{figure}[ht]
    \centering
    \includegraphics[width=17cm]{images/istatik.png}
    \caption{Istatik Yönlendirme}
    \label{fig:bandwidth_example}
\end{figure}
\subsection{DİNAMİK YÖNLENDİRME}
Yönlendirme algoritmaları  tarafından hesaplanarak bulunur.Ağ yöneticisi tarafından önceden bazı filtreler ve tanımlamalar yapılmalıdır.Ağda değişiklik olduğundan yollar otamatik olarak düzeltilir.En yaygın yönlendirme algoritmadir.\\
\begin{figure}[ht]
    \centering
    \includegraphics[width=17cm]{images/dinamik.png}
    \caption{Dinamik Yönlendirme}
    \label{fig:bandwidth_example}
\end{figure}

şekildeki ağın sağlıklı çalışabılmesi için  ..... sağlanmalıdır\\
\begin{figure}[ht]
    \centering
    \includegraphics[width=17cm]{images/dinamik1.png}
    \caption{Dinamik Yönlendirme}
    \label{fig:bandwidth_example}
\end{figure}
\begin{enumerate}
   \item C'nin E1 bacağı ile A aynı ağda olmalıdır.
   \item  C'nin E2 bacağı ile B aynı ağda olmalıdır.
   \item  A'nin ağ gecıdı C'NIN E1 bacağındakı IP olmadır.
   \item  B'nın ağ gecidi C'nın E2 bacağındakı IP olmalıdır.
   \item C'ye IP yönlendirme komutu verilmelidir.
\end{enumerate}

Yönlendirme  tablosunda  birbirini kapsayan kurallar var ise bunlar küçükten büyüğe sira ie değerlendirilir.\\

\begin{figure}[ht]
    \centering
    \includegraphics[width=17cm]{images/ip.png}
    \caption{Dinamik Yönlendirme}
    \label{fig:bandwidth_example}
\end{figure}

örnek trafik 10.0.0.5 IP'yi google götürmek üzere /26 kulanır . Gerçi hepsini kapsıyor ondan en küçüğün kabul eder . 10.0.0.199 google göstermek için /24 kullanır.\\

\begin{figure}[ht]
    \centering
    \includegraphics[width=17cm]{images/switch.png}
    \caption{Dinamik Yönlendirme}
    \label{fig:bandwidth_example}
\end{figure}

\begin{center}
 \underline{Yönlendirme Tablosu}
\end{center}
\begin{table}[h]
   \centering
   
   \begin{tabular}{llll}
   A $\rightarrow$ B & 10.0.1.0/24 & 10.0.2.0/24 & E2\\
   B $\rightarrow$ A & 10.0.2.0/24 & 10.0.1.0/24 & E1\\
   (A+B)    $\rightarrow$  & 0.0.0.0/0 & 0.0.0.0/0 & E3
  \end{tabular}
  \caption{}    
\end{table}

\textcolor{red}{NOT : } Yönlendiriciler de kendisine doğrudan bağlıdan (directly connecte) ağlar için genellikle yönlendirme  çünkü doğrudan bağlı olan bütün ağlar tanırlar.

\begin{table}[h]
   \centering
   \begin{tabular}{lll}
   \textcolor{red}{\underline{Directly Connected}}&   
   & \textcolor{red}{\underline{Eklenmesi Gerekenler}}\\
   A$\rightarrow$ 1,2 && A$\rightarrow$ 3,5,b \} 3 satır kural eklemesi gerekiyor\\
   B$\rightarrow$ 2,3,4 && A$\rightarrow$ 1,5,b \} 3 satır\\
   c$\rightarrow$ 1,2 &&A$\rightarrow$ 1,3 \} 2 satır \\
  \end{tabular}
\end{table}
