\section{IP YÖNLENDİRME}

Ip'nın yönlendirilebilir olmasi protokolün  en güçlü özelliği çok sayıda iletişim protokolu mevcut olmasına rağmen IP'nin yönlendirilebilir esnek yapısı internetin temel dili olmasını sağlamıştır.Yönlednirme işlemini "Yönlendirici(Router) "yapar.
\textbf{Yönlendirme Tablosunda;}
*Kaynak \\
*Hedef(Ip ve Maske) \\
*Ağ Geçidi\\
*Ara Birim(Interface)\\
*Ölçüt(Metrik)\\
\subsection{1-STATİK YÖNLENDİRME}
Ağ yöneticisi tarafından elle sabit olarak yazılır.Genellikle yönlendiricisi ve yönlendirme işlemi çok fazla olmayan ağlarda kullanılır.Yönlendirme tablolarının güncellemesi ağdaki fiziksel değişikliklere göre yeniden elle yapılmalıdır.
\subsection{2-DİNAMİK YÖNLENDİRME}
Yönlendirme algoritmaları  tarafından hesaplanarak bulunur.Ağ yöneticisi tarafından önceden bazı filtreler ve tanımlamalar yapılmalıdır.Ağda değişiklik oldu















2ni
