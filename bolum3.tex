\section{İLETİM ORTAMLARI}
Temelde atmosfer ve kablo olmak üzere iki farklı iletim ortamı mevcuttur.Atmosfer rf(radyo frekans) dalgalarını kullanarak  iletişim gerçekleşir.
Kablolarda ise genellikle fiberoptik ve bakır kablo kullanılmaktadır.

\subsection{İKİ TELLİ BAKIR TELEFON HATTI }
Telefon iletişimini sağlamak için tasarlanmıştır.Temel band ve geniş band internet hizmeti verilmektedir.
Analog modülasyon teknikleryle en fazla 56 k b/s'lik band genişliği sağlar.xDSL teknolojileriyle 25 M b/s'lik band genişliğine ulaşmaktadır.  
\begin{figure}[!ht]
    \includegraphics{images/ikitellibakirkablo}
   \caption{İki telli Bakir Kablo}
    \label{fig:iki_telli_bakir_kablo}
 \end{figure}

 \subsection{KOOKSİYEL(CCOKSİAL)KABLO}
 Genellikle elektiriksel gürültünün yoğun olduğu şartlarda kullanılırdı.Yalıtkan bir tüpün içerisinde  giden bir tel ve tüpün dışına sarılmış kafes şeklinde teller vardır.
 Yerel ağlarda (LAN) 180m'de(max) 10M b/s bant genişliği sağlar.Bu kullanımı 10 Base 2 olarak bilinir.Daha sonra 500 m mesafede çalıştırılacak hale getirilir. 10 Base 2 ismiyle standartlaştırılmıştır.50 ohm'luk direnç değeri vardır.
 BNC tarzında connectörler kullanılır.Günümüzde LAN'da hiç kullanılmamaktadır.Sebebi hem 1o M b/s  hızının çok düşük olması,hem de UTP kablolar kadar ekonomik ve işlevsel olmamasıdır.
 Bilgisayr ağlarında doğrusal (bus) topolojilerde kullanılmıştır. 
 
  \begin{figure}[!ht]
     \includegraphics{images/400px-RG-59}
    \caption{Kooksiyel Kablo}
     \label{fig:kooksiyel_kablo}
  \end{figure}


 \section*{AĞ TOPOLOJİLERİ }
\begin{figure}[!ht]
    \includegraphics{images/62_ağ_topolojisi}
   \caption{Topolojiler}
    \label{fig:topolojiler}
 \end{figure}
 % bu kısm notlarda yok topoloji nedir kısmını ben ekledim.
Ağ topolojileri nedir sorusunun en net cevabı, “bir ağı oluşturan cihazların fiziksel ve mantıksal yerleşimidir“.Network Topology (Ağ Topolojisi) Yerel Ağ Alanı (LAN) içerisinde bulunan bilgisayarların fiziksel ve mantıksal yerleşimini ifade eder. Fiziksel Topoloji ağ içerisinde bulunan tüm cihazların birbirlerine nasıl bağlanacağını ve bağlantı için ne tür kablo kullanacığını belirtirken Mantıksal Topoloji bu cihazların nasıl haberleşeceğini belirtir ve bu cihazları ortak bir protokol altında birleştirir. Kullanılmak istenen Ağ Teknolojisine göre farklı ağ topolojileri kullanılmaktadır.
Fiziksel Topolojinin 6 farklı çeşidi vardır. Bunlar Bus(Yol), Ring(Halka), Yıldız(Star), Ext Star(Gelişmiş Yıldız), Mesh(Örgü) ve Tree(Ağaç) topolojileridir. Broadcast(Yayın) ve Token Passing(İz) mantıksal topolojilere birer örnektir 

\subsection*{DOĞRUSAL (BUS) TOPOLOJİ}
Doğrusal bir hat üzerinde bilgisayarların T konnektörlerle bağlanması şeklinde kurulur.Hattın her iki ucunda 
sonlandırıcı kullanmak zorunludur.Kooksiyel kablo kullanılır.Ağın herhangi  bir noktasında arıza olması durumunda ağın tamamı çöker.Ağdaki veri trafiği tüm uçlara gider.
Herkes herkesin trafiğini görebilir.Bu yüzden çok fazla \textbf(çakışma(cooolision)) olur.
\begin{figure}[!ht]
    \includegraphics{images/bus-topolojisi}
   \caption{Bus Topolojisi}
   \label{fig:bustopolojisi}
 \end{figure}

\subsection*{HALKA (RING) TOPOLOJİ}
Doğrusal topolojiye benzer.Sonlandırıcı kullanılmaz.Hattın iki ucu birleşiktir.
Hatta sanal  bir jeton dolaşır(token).Jeton sırası gelen bilgisayar,jeton boş ise göndereceği veriyi hatta yerleştirir.
Bilgisayarlar sırayla  veri gönderdiklerinden çakışma daha azdır.Günümüzde hiç kullanılmamaktadır.
Herkes herkesin verisini kullanabilmektedir.
%bu fotoğraf en alta geçiyor düzeltilecek
\begin{figure}[!ht]
    \includegraphics{images/ring-topology-removebg-preview}
    \caption{Halka-Ring Topolojisi}
   \label{fig:halka_topolojisi}
  \end{figure}


\subsection*{YILDIZ (STAR) TOPOLOJİ}
Merkezde dağıtıcı bir cihaz olur.Burdan tüm bilgisayarlara birer kablo gider.Ağın bir noktasındaki arıza sadece ilgili bilgisayarın ağ bağlantısına zarar verir.Genellikle \textbf(bükümlü çift (twisted pair,xtp)) kullanılır.Trafiğin herkese mi gönderileceği ya da sadece ilgili ucamı gideceği dağıtıcıya bağlıdır.
Dağıtıcının  performansı ve kabiliyeti ağı doğrudan  etkiler.
Günümüzde en yaygın topolojidir.
%YERİ DÜZELTİLECEK
\begin{figure}[!ht]
    \includegraphics{images/star-Topology-1024x512-removebg-preview}
    \caption{Yildiz-StarTopolojisi}
   \label{fig:yildiz_topolojisi}
  \end{figure}

\subsection*{ÖRGÜ (MESH)TOPOLOJİ}
%YERİ DÜZELTİLECEK
\begin{figure}[!ht]
 \includegraphics{images/mesh-topology-1-1024x512-removebg-preview}
    \caption{Örgü-Mesh Topolojisi}
   \label{fig:Orgu_mesh_topolojisi}
  \end{figure}
Uçları arasında birden fazla rota üzerinde haberleşme imkanı olan yapılardır.
Günümüzde genellikle farklı yıldız ağlar arasında yedekleme amacı olarak kullanılır.


\subsection{BÜKÜMLÜ ÇİFT KABLO}
  İçerisinde 4 çift bakır kablo bulunur.Kabloların birbireri üzerindeki direnç elektromanyetik etkisini azaltmak için ikişerli olarak sarılı durumundadırlar.
  Örneğin; UTP,CAT5,Ethernet Kablosu 
  \begin{figure}[!ht]
   \includegraphics{images/bukumlukablo}
   \caption{Bükümlü çift kablodan bir kesit}
 \label{fig:Bukumlu_cift_kablo}
 \end{figure}

\subsubsection{UTP(UNSHİLDED PWİSTED PAİR)Korumasız Bükümlü Çift} 
    8 iletkenin her biri ince bir yalıtkan ile kaplanmıştır.En dışında tamamını kaplayan bir yalıtkan vardır.
\subsubsection{STP(SHİLDED TWİSTED PAİR)} 
Her çiftin altında koruma (topraklama ) vardır.
\subsubsection{FTP(FOİLED TWİSTED PAİR )} 
4 çiftin tamamının etrafında folyo koruma vardır.
\subsubsection {S/FTP }
İkisinin de özelliğini taşımaktadır.
    
\subsection{FREKANSLARINA GÖRE BÜKÜMLÜ ÇİFT KABLO}
\textbf{CAT:}\\
\textbf{CAT1-CAT3} \\
Telefon hatlarında bulunur.\\
\textbf{CAT5} \\
En yaygın kullanılan ağ kablosudur.Azami 100 m mesafe ve 10 m b/s destekler\\
\textbf{CAT6} \\
100 m mesafede 1G b/s destekler.\\
\textit{10 BASE T} Ethernet(Eth)\\
\textit{100 BASE T} Fast Ethernet(Fa,Fe)\\
\textit{1000 BASE T} Gigabit Ethernet(G,GE)\\
Bükümlü çift CAT5 VE CAT6 Kabloları  sonlandırmak için RJ-45 adı verilen connektörler kullanılır.
Bu kablolar iki farklı iki şekilde sonlandırılabilir.\textbf{568-A,568-B}\\
Kablonun iki ucunun aynı standarlarla sonlandırılmasna \textbf{Düz(Straight kablo )} denir.İki ucunda iki farklı standartta sonlandırılma yapılırsa \textbf{çapraz(cross-over)kablo } adı verilir.
\subsection{ÇAPRAZ VE DÜZ KABLO}
Düz kablo, bir bilgisayarı yönlendirici gibi bir ağ hub'ına bağlamak için yerel alan ağlarında kullanılan bir tür bükümlü çift kablodur. Bu tür kablolara bazen yama kablosu da denir ve bir veya daha fazla bilgisayarın kablosuz bir sinyal yoluyla bir yönlendiriciye eriştiği kablosuz bağlantılara bir alternatiftir.
Aynı türden iki cihazı bağlamak için genellikle bir çapraz kablo kullanılır.Düz kablo ve çapraz kablo tasarımları aynı standartların ve kuralların çoğunu kullanır.

\begin{figure}[!ht]
  \includegraphics{images/caprazduzz}
 \caption{kablolar}
 \label{fig:caprazduz_kablo}
\end{figure}

Yeni ağ cihazlarının tamaı MDI/MDIX adı verilen teknoloji sayesinde karşıdaki cihazın ne tarz bir cihaz olduğunu anlar ve hangi iletkenin ne amaçla kullanılacağını buna göre  düzenler. 
Diğerleri enerji göndermek için kullanılır.\\

\begin{figure}[!ht]
  \includegraphics{images/ethcable}
 \caption{kablolar-ornek}
 \label{fig:caprazduz_kablo_ornek_gosterim}
\end{figure}
\section*{FİBER OPTİK KABLOLAR}
Fiber optik kablolar, veri göndermek için ışık sinyallerini kullanmaktadır. Bu kablolar elektrik kablolarına benzer. Ancak elektrik kablolarından farklı olarak ışığı taşımak için kullanılan minimum bir adet fiber optik içeren bir kablo çeşididir.\\
Fiber optik kablolar çeşitli özelliklere ve avantajlara sahiptirler. Fiber optik kablonun farklı alanlarda  bu kadar sık tercih edilmesinin nedenleri kabloların bulundurduğu özellikler ve sunduğu bu avantajlardır. \\
\textbf{Fiber Optik Avantajları}\\
Elektrik parazitlerinden etkilenmez.\\
Sıcaklık değişimleri ve neme karşı dayanıklıdır.\\
Metalik kablolardan daha hafif ve daha küçüktürler.\\
Sinyal kaybı yok denecek kadar azdır ve sinyal güçlendirici ihtiyacını azaltır.\\
Sıcaklık değişimleri, su baskınları, şiddetli hava ve nem gibi çevresel parametrelere karşı dayanıklıdır.\\
Bu kablolarda iletim için ışığın yansımasını kullanılır. Böylelikle bu kablolar çok daha uzun mesafelere veri iletimi yapılabilirler.
Elektromanyetik enerji sızması meydana gelmediği için bilgi güvenliği sağlanmış olur.\\
Bu kablolar ile bilginin ekonomik, verimli ve hızlı bir şekilde ulaştırılması sağlanır.\\
\textbf{Fiber Optik Dezavantajları}\\
Sınırlı Uygulama — Fiber optik kablo sadece zeminde kullanılabilir ve zemini terk edemez veya mobil iletişim ile çalışamaz.\\
Düşük Güç — Işık yayan kaynaklar, düşük güçle sınırlıdır. Güç kaynağını iyileştirmek için yüksek güç yayıcıları bulunmasına rağmen, ek maliyet ekleyecektir.\\
Kırılganlık - Fiber optik, bakır tellere kıyasla daha kırılgandır ve hasara karşı daha hassastır. Fiber optik kabloları bükmemeli veya bükmemelisiniz.\\
Mesafe — Verici ve alıcı arasındaki mesafe kısa olmalı veya sinyali arttırmak için tekrarlayıcılara ihtiyaç duyulmalıdır.\\

%\begin{figure}[!ht]
 % \includegraphics{images/fiberoptik}
 %\caption{fiberkablo}
 %\label{fig:fiberoptikkablo}
%\end{figure}

 Veri optik dalgalar arıcılığı ile ışığın yansıma kurallarına göre elde edilir.
 Elektiriksel sinyallerine göre mesafeye bağı zayıflama sinyalleri çok azdır.
 Bakır kablolarda olduğu gibi gerilim farkından kaynaklanan topraklama ihtiyacı yoktur.
 Fiberoptik kabloların yerel ağa bağlanmasında elektiriksel sinyal ile optik dalgalar arasında çevrilmesi gerekir.Verici tarafından ışık kaynağı olarak lazer diyod(led),alıcı tarafında ise fotodiyod ya da foto transistör kullanılır.
\subsection{FİBER OPTİK KABLO TÜRLERİ}
Single mod(SM) ve Mlti mod(MM) olmak üzere ikiye ayrılır\\
\textbf{Multi-Mode}\\
Bina ya da kampüs içi kısa mesafelerde tercih edilir.Optik dalga üretmek için Led kullanılır.Verici ve alıcı maliyetleri single moduna göre yarı yarıya azdır.
\textbf{Single-Mode}
Hem daha uzun mesafelerde hem de daha yüksek band genişliğine imkan sağlar.
Optik dalga üretmek için LazerDiyod kullanılır.Bu nedenle verici ve alıcı donanım maliyetleri daha fazladır.

\begin{figure}[!ht]
  \includegraphics{images/single-multimode}
 \caption{single-multimode}
 \label{fig:single_multi_mode}
\end{figure}
\subsection*{FİBER OPTİK ÇEVİRİCİLER}
*F/O CONVERTOR \\
*F/O TRANSREİVER (ALICI/VERİCİ)\\
*GBIC  (Switch modülü halindedir) \\
*STP  (Switch modülü halindedir) \\

\subsection*{YEREL AĞLAR (LAN)}
Kablo çekebileceğimiz (bize ait  olan ) yerler yerel ağlardır.
Ağlarda band genişliği ,protokol,topoloji gibi altarnetifler isteğe göre özelleştirilebilir.
Günümüz yerel ağlarında ethernet harici protokol kullanılmamamktadır.
\subsection*{ETHERNET PROTOKOLÜ}
İlk kez "INTEL VE XEROX" tarafından geliştirilmiştir.Daha sonra IEEE(Institue of Electrical and Electronical Enginner) tarafından 809.3 ismi ile standartlaştırılmıştır.
\subsection*{10 M b/s Ethernet Portları}
\textbf{10 Base 2 :} 10 sayısı 10 m b/s'yi ifade eder.Base sözcüğü temel bandı ifade eder.En sondaki kablo türüdür.2 olduğunda ince (thin) kooksiyel kablodur.\\
\textbf{10 Base 5 :} Sondaki 5 Kalın(thick) kooksiyel kablo olduğunu belirtir.\\
\textbf{10  Base T :}Bükümlü çift kablo olduğunu ifade eder.\\
\subsection*{100 M b/s ETHERNET PORTLARI}
\textbf{100 Base Tx :}Fast Ethernet Cat-5 kablo kullanılır.\\
\textbf{100 Base Fx :}F harfi Fiberoptik Kablo kullanıldığını belirtir.\\
\subsection*{1000 M b/s ETHERNET PORTLARI}
\textbf{1000 Base-T :} Cat5 ve Cat6 kablolar kullanılır.Ancak Cat6 tercih edilir.\\
\textbf{1000 Base-Lx :} L long kısaltmasıdır.SM,MM,FO kblolar kullanılır.Uzak mesafelerde tercih edilir.En önemli dezavantajı maliyetlerin SX'e göre fazla olmasıdır.\\
\textbf{1000 Base-SX :}Yalnızca mm FO kablolar kullanılır.Kısa mesafeleri destekler.Ekipmanları daha ucuzdur.\\
\subsection*{FİBEROPTİK SONLANDIRMA ŞEKİLLERİ}
LC,SC,ST,FC sonlandırma mevcuttur.Günümüzde en yagın olan "LC" tipi sonlandırma şeklidir.\\
\begin{figure}[!ht]
  \includegraphics{images/Fiber-Optik-Ek-ve-Sonlandirma}
 \caption{fibersonlandırma}
 \label{fig:fosonlandırma}
\end{figure}
\\
*F/O eki füzyon cihazı ile yapılır.2 tane cam tüpleri kaynatarak birbirine ekler.\\
İşlemler mikron seviyesinde yapıldığından kendi mikroskopu olan ve hassasiyeti yüksek olan cihazlar kullanılır. \\
F/O kablo testleri "OTDR" isimli cihaz  ile yapılır.
