\section{OSI MODELİ (OSI KATMANLARI)}
Bir bilgisayardan gönderilen bir bilginin diğer bilgisayara nasıl ulaştığını anlatmak için tasarlanmıştır.
İletişimi 7 katmanlı mimarı ile tanımlar. Ağ elemanlarının nasıl çalıştığını ve verinin iletimi sırasında hangi işlemlerden gectiğini kavramak için kullanılan rehberdir.
OSI Katmanlarının mantığını anlamak ağları planlamak, ağ üzerinden çalışan program yazmak ve ağ sorunlarını çözmek için önemlidir.
\subsection{Katmanlar}
\begin{enumerate}
	\item Fiziksel (Physical)
	\item Veri Bağı (Data link)
	\item Ağ (IP)
	\item Taşıma (Transport)
	\item Oturum (Session)
	\item Sunum (Presentation)
	\item Uygulama (Application)
\end{enumerate}

\subsubsection{Fiziksel Katmanlar}
Haberleşme kanalının elektriksel ve mekanik olarak tanımlandığı katmandır. Bir uçten gönderilen sinyalin karşı uca iletilmesinden sorumludur. Sayısal haberleşmede en küçük birim bit olduğundan bu katmanın hızı \bf{(bps) (b/s) bit/saniye} cinsindendir.
Birinci katman donanımları:
\begin{enumerate}
	\item Bakır ve fiber optik kablolar
	\item RF (Antenler)
	\item Sinyali(işareti) elektrik olarak yükselten ve çoklayan HUB cihazları
	\item Kablosuz iletişimde kullanılan hava
\end{enumerate}

\subsubsection{Veri Bağı Katmanı}
Verinin fiziksel ortamdan güvenli bir şekilde taşınmasından sorumlu olan katmandır. Kaynaktan çıkan verilerin(bitler) hedefe ulaşan verilerle aynı olup olmadığını sınayan sistemler kullanılır.
En çok kullanılan hata bulma algoritmaları \textbf{eşlik biti(perity check)} ve \textbf{CRC algoritmasıdır}. Verinin doğru olup olmadığına bakmaz, sadece sağlamlığını kontol eder.
Bu katmanda üst katmandan gelen veriler ???? adı verilen paketleme işlemini tabi tutulur. Kapsülleme de denir. Birbirine doğrudan bağlı ağ cihazlarının aynı kapsulleme yöntemini(ikinci katman protokolünü) kullanması gerekir.

\begin{table}[h]
	\centering
	\caption{Kapsülleme}
	\label{tab:table_kapsulleme}
	\begin{tabular}{|c|c|c|}
		\hline
		Kaynak & Veri & ??? \\
		\hline
	\end{tabular}
\end{table}

\subsubsection*{Günümüzde en yaygın ikinci katman protokolleri}
\textbf{Yerel ağda (LAN)} : Ethernet \\
\textbf{Uzak ağlarda (WAN)} : AIM, PPP, Frame, Relay, Metroethernet
\subsection*{Anahtarlama}
\begin{enumerate}
	\item[$\blacksquare$] \textbf{Devre Anahtarlama}: Veri aktarımı, fiziksel değişiklikle yapılır.
	\item[$\blacksquare$] \textbf{Paket Anahtarlama}: Veri aktarımı, her bir veri paketi için hesaplanarak, yazılımsal olarak yapılır.
\end{enumerate}
Ethernet protokolünde kaynak ve hedef adresleri olarak MAC adresi (fiziksel adresi) kullanılır.\\
	Çakışmaları engellemek için aynı ağda iki MAC adresi olmamalıdır.

Anahtarlar(switch) bu katmanda calışır. Anahtarlar portlarına bağlı olan cihazların MAC adreslerini bilmek zorundadır(otomatik öğrenir). Bu şekilde iki farklı portu arasındaki trafiği diğer cihazlar görmeden aktarabilirler \textbf{HUB'lardan en önemli farkı budur}.
\subsubsection{AĞ Katmanı (IP) }

\subsubsection{Taşıma Katmanı}

\subsubsection{Uygulama Seviyesi Katmanları}

