\section{OSI MODELİ (OSI KATMANLARI)}
Bir bilgisayardan gönderilen bir bilgini diğer bilgisayara nasıl ulaştığnı anlatmak için tasarlanmıştır.
İletişim 7 katmanlı mimarı ile tanımlar Ağ elemanlarının nasıl çalıştığını ve serinin iletimi sırasında hangi isimlerden gectiğini kavramak için kullanılan rehberdir.
OSI Katmanlarının mantığını anlatmak ağları planlamak, ağ üzerinden çalısan program yazmak ve ağ sorunların çözmek için önemlidir.
\subsection{Katmanlar}
\begin{enumerate}
	\item Fiziksel (Physcal)
	\item Veri Bağı (Data link)
	\item Ağ (IP)
	\item Taşıma (Transport)
	\item Oturum (Session)
	\item Sunum (Presentation)
	\item Uygulama (Application)
\end{enumerate}

\subsubsection{Fiziksel Katmanlar}
Haberleşme kanalının elektriksel ve mekanik olarak tanımlandığı katmandır. Bir uçten gönderilen sinyalin karşi uca iletilmesinden sorumludur. Sayısal Haberleşmede en küçük birim bit olduğundan bu Katman hızı \bf{(bps) (b/s) bit/saniye} cinsindendir.
Birinci katman donanımları:
\begin{enumerate}
	\item Bakır ve FiberOptik Kablolar
	\item RF (Antenler)
	\item Sinyali
	\item Kablosuz iletişimde kullanılan Hava
\end{enumerate}

\subsubsection{Veri Bağı Katmanı}
Verinin fiziksel ortamdan güvenli bir şekilde taşınmasından sorumlu olan katmandır, kaynaktan cıkan verilerin(bitler) hedef ulaşan verilerle aynı olup olmadığını sınayan sistemler kullanılır.
En çok kullanılan hata bulma algoritmalar \textbf{eşlik biti(perity check)} ve \textbf{CRC algoritmasıdır}. Verinin doğru olup olmadığı bakmaz, sadece sağlamlığını kontol eder.\
Bu katmanda üst katmandan gelen veriler ???? adi verilen paketleme işlemini tabi tutulur. ??? de denir. Birbirine doğrudan bağlı ağ cihazlarının aynı kapsulleme yöntemini ikinci katman protokolünü kullanması gerekir.

\begin{table}[h]
	\centering
	\caption{Kapsülleme}
	\label{tab:table_kapsulleme}
	\begin{tabular}{|c|c|c|}
		\hline
		Kaynak & Veri & ??? \\
		\hline
	\end{tabular}
\end{table}

\subsubsection*{Günümüzde en yaygin ikinci katman protokolleri}
\textbf{Yerel ağda (LAN)} : Ethernet \\
\textbf{Uzak ağlarda (WAN)} : AIM, PPP, Frame, Relay, Metroethernet
\subsection*{Anahtarlar}
\begin{enumerate}
	\item[$\blacksquare$] \textbf{Devre Anahtarlama}: Veri aktarımı, fiziksel değişiklikler yapılır.
	\item[$\blacksquare$] \textbf{Paket Anahtarlama}: Veri aktarımı, her bir veri paketi için hesaplanarak, yazılımsal olarak yapılır
\end{enumerate}
Ethernet protokolünde kaynak ve hedef adresleri olarak MAC adresi (fiziksel adresi) kullanılır.\\
	Aynı ağda iki MAC adresi olmamalı sebebi çakışmalar engellemek için.

Anahtarlar(switch) bu katmanda calışır. Anahtarlar portlarına bağlı olan cihazların MAC adreslerini bilmek zorundadır( otomatık öğrenir). Bu şekilde iki farklıı portu arasındakı trafığı diğer cıihazlar görmeden aktarabilirler \textbf{HUB'lardan en önemli farkı budur}.
\subsubsection{AĞ Katmanı (IP) }

\subsubsection{Taşıma Katmanı}

\subsubsection{Uygulama Seviyesi Katmanları}

