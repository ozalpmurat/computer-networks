\section{OSI MODELİ (OSI KATMANLARI)}
Bir bilsigarden gönderilen bir bilgini diğer bilgisayara nasıl ulaştığnı anlatmak için tasarlanmıştır.
İletişim 7 katmanlı mimarı ile tanımlar Ağ elemanlarının nasıl çalıştığını ve serinin iletimi sırasında hangi isimlerden gectiğini kavramak için kullanılan rehberdir.
OSI Katmanlarının mantığını anlatmak ağları planlamak, ağ üzerinden çalısan program yazmak ve ağ sorunların çözmek için önemlidir.
\subsection{Katmanlar}
\begin{enumerate}
	\item Fiziksel (Physcal)
	\item Veri Bağı (Data link)
	\item Ağ (IP)
	\item Taşıma (Transport)
	\item Oturum (Session)
	\item Sunum (Presentation)
	\item Uygulama (Application)
\end{enumerate}

\subsubsection{Fiziksel Katmanlar}
Haberleşöme kanalının elektriksel ve mekanik olarak tanımlandığı katmandır. Bir uçten gönderilen sinyalin karşi uca iletilmesinden sorumludur. Sayısal Haberleşmede en küçük birim bit olduğundan bu Katman hızı \bf{(bps) (b/s) bit/saniye} cinsindendir.
Birinci katman donanımları:
\begin{enumerate}
	\item Bakır ve FiberOptik Kablolar
	\item RF (Antenler)
	\item Sınyali
	\item Kablosuz iletişimde kullanlan Hava
\end{enumerate}

\subsubsection{Veri Bağı Katmanı}


\subsubsection{AĞ Katmanı (IP) }

\subsubsection{Taşıma Katmanı}

\subsubsection{Uygulama Seviyesi Katmanları}

